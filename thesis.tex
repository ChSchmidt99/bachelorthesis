\documentclass[11pt,a4paper]{article}
\usepackage{theme/lmuthesis}
\usepackage{glossaries}

% this for draft water mark
\input{condition}
\ifthenelse{\boolean{release}}{
}{
    \usepackage{draftwatermark}
    \SetWatermarkText{DRAFT}
    \SetWatermarkScale{1}
}

% meta informations
\department{Institut f\"ur Informatik}
\lfe{Lehr- und Forschungseinheit Medieninformatik}
\professor{Prof.\ Dr.\ Butz}
\type{Bachelor's Thesis}
\title{Progressive BVH Refinement in Interactive Ray Tracing}
\author{Christian Schmidt}
\email{nichtchristianschmidt@gmail.com}
\bearbeitungszeitraum{06.05.2021 bis END}
\supervisor{Changkun Ou}
\taskdescription{
    \begin{description}
        \item[Progressive BVH Refinement in Interactive Ray Tracing]
        \item[Problem Statement] Path tracing in real time become avaliable on the GPU side in recent years due to the recent advances in image denoising techniques, such as NVIDIA DLSS.

        It is interesting to optimize and render everything directly on the CPU in real time for reasonably smaller scenes.
        \item[Tasks]
        - Implement a multi-threaded CPU ray tracer
        - Profile and identify the bottleneck of your ray tracer implementation
        - Benchmark and compare the performance difference between your CPU ray tracer and an equivalent CUDA ray tracer
        - Summarize your findings in a thesis and presenting them to audiences
        \item[Requirements]
        - Have experience or projects using C++ (or Go)
        - General knowledge about computer graphics
    \end{description}
}
\acknoledgement{
    I would like to appreciate ...
    % TODO: Thank Changkun for help and provided frameworks
}
\abstract{
    This thesis proposes ...
}

\begin{document}

\makecover
\maketaskdescription
\makededication
\makeabstract
\maketoc
% optional:
% \listoffigures
% \listoftables
\cleardoublepage
\section{Introduction}
Ray tracing as a rendering technique simulates how light travels through a scene and thus inherently produces very realistic images. Effects that need to be computed explicitly in the contrasting approach of rasterization such as shadows, reflections and refractions are achieved by default albeit introducing a substantial computational cost. It is this simplicity paired with the high quality output, that made the method a staple in offline rendering where relatively long rendering times can be tolerated.

In real-time applications the time available to render a single frame is very limited making ray tracing a rather poor fit and leaving rasterization as the dominant approach over the past decades. Only in recent years has real-time ray tracing been enabled and opend up to a consumer market, especially through the use of special-purpose hardware. A particularly noteworthy milestone in that regard is NVIDIA's Turing architecture\cite{nvidia2017turing}, which is built in a way that accelerates basic ray tracing operations while also facilitating other methods essential to the process, most importantly more advanced denoising techniques. 

While this technology is undoubtedly a leap in the right direction, performance can still lack at times and additional software optimizations are necessary to support increasingly complex scenes. Additionally, such optimizations could decrease the power consumtion of graphic cards by decreasing the number of computations necessary and could be utilized by conventional ray tracing frameworks as well. Consequently, the search for better algorithms to accelerate ray tracing is still an interesting topic and was the main motivation behind this thesis. 

% TODO: Add references to the corresponding sections
In particular, this thesis provides a self-contained overview over the basics of real-time ray tracing as well as a broader outline over some related topics to make it as accessible as possible. All findings are based on an interactive CPU path tracer written from scratch in the programming language Go, which is elaborated further including some implementation details. This might be particularly helpful as a starting point for further research and optimizations. Furthermore, a closer look at bounding volume hierarchies is presented, and the approach of progressive hierarchical refinement\cite{hendrich_parallel_2017} is introduced as a state-of-the-art algorithm to construct such acceleration structures. The integration of the aforementioned approach into an interactive path tracer is described and validated, as this was still an open topic. Finally, two methods for optimizing hyperparameters related to the construction algorithm in real-time are introduced, evaluated and discussed.
\section{Preliminaries}
This section introduces the basic concepts needed to follow along with the rest of this thesis, while also putting some focus on notable historical work.
\subsection{Ray Tracing}
\label{path_tracing_basics}
\begin{figure}[H]
    \centering
    \includegraphics[width=250pt]{images/ray_tracing.pdf}
    \caption{Illustration of the path tracing process. A ray is cast into the scene for each pixel in the image plane. When an object is hit, its color is contributed to the pixel and a new ray is spawned.}
    \label{fig:path_trace}
\end{figure}
The basic technique of casting rays through each pixel in a viewing plane out into a three dimensional scene, was first proposed by Appel\cite{appel1968}. Whitted\cite{whitted_improved_1980} expanded on that approach by introducing a method that also captures reflections, shadows and refractions. This works by recursively spawning new rays when objects are hit, as described in figure \ref{fig:path_trace}. Each consecutive hit can contribute to a part of the pixels color. Whether and in what fashion secondary rays are created depends on the material of the encountered object. 

\begin{figure}[H]
    \centering
    \subcaptionbox{1 spp}{\includegraphics[width=0.2\textwidth]{images/cornell_1_spp.png}}
    \hfill
    \subcaptionbox{10 spp}{\includegraphics[width=0.2\textwidth]{images/cornell_10_spp.png}}
    \hfill
    \subcaptionbox{100 spp}{\includegraphics[width=0.2\textwidth]{images/cornell_100_spp.png}}
    \hfill
    \subcaptionbox{1000 spp}{\includegraphics[width=0.2\textwidth]{images/cornell_1000_spp.png}}
    \caption{Noise at different sample-per-pixel (spp) values. Note that the images were rendered with implicit light sources. Noise is reduced using explicit light rays.}
    \label{fig:noise}
\end{figure}

The highest amount of realism can be achieved by a related concept called path tracing\cite{cook_distributed_1984, kajiya_rendering_1986}, which enables the accurate rendering of global illumination and distribution effects. Instead of casting a single ray per pixel, multiple samples are combined to more accurately simulate light transport through a scene. Each ray is cast with a slight offset within the pixel, which inherently solves the problem of aliasing. Secondary rays are scattered in random directions, depending on the surface they hit. Provided a sufficiently large sample size, this allows for a good approximation of the rendering equation\cite{kajiya_rendering_1986} and leads to very realistic results. Consequently, path tracing is very common in offline rendering, e.g., in film\cite{keller2015path_tracing_revolution} and visual effects, where relatively long render times can be tolerated. Given the limited time to process a frame in real-time, however, only allows to obtain a very limited number of samples. As seen in figure \ref{fig:noise}, 
these frames tend to suffer from high-frequency noise. Only through recent denoising techniques, further discussed in section \ref{denoising}, has this problem been overcome to make path tracing viable in real-time. A more detailed look at path tracing is presented in section \ref{path_tracer}.
\subsection{Acceleration Structures}
\label{acceleration_structure_basics}
\begin{figure}[H]
    \centering
    \includegraphics[width=250pt]{images/bvh_kd_tree.pdf}
    \caption{Example of space subdivision (left) and object subdivision (right).}
    \label{fig:subdiv}
\end{figure}
The essential and computationally most expensive step in path tracing is identifying the nearest intersection point for each ray. A naive approach would be to test the ray against all scene primitives, which in practice might be several million operations and thus too costly. Acceleration data structures are used to speed up that search process by arranging primitives in a spatial tree structure. Instead of checking all primitives, now the tree can be traversed to find the closest intersection. While doing so, all subtrees that are not hit by the ray can be neglected, reducing the number of total intersection tests drastically. Such data structures can be divided into two categories, space subdivision and object subdivision (figure \ref{fig:subdiv}). 

Space subdivision works by splitting the scene space recursively into smaller subregions, so each leaf in the resulting tree corresponds to a disjoint area in the full scene. This makes traversal of those structures very efficient, as the subregions can be tested in the order any ray passes through them. If a intersection is found, the traversal algorithm can be terminated without needing to check any further nodes. One of the limitations of such approaches is, that an object might lie in multiple subregions at once, i.e. multiple leaves might contain a pointer to the same object. This is problematic, because that way a point lying outside the associated region might be falsely identified as an intersection. The traversal algorithm as described would assume that point as the closest hit, even though there is no guarantee for that. Furthermore, this way the same object might be tested multiple times. Objects can be clipped to solve this problem, however, this introduces a certain computational cost. Alternatively, the traversal algorithm can be extended to also check whether or not a intersection point lies within a node's region. 

Object subdivision leads to trees with similar structure, however, each node is also associated with a bounding volume. Bounding volumes are a geometric shapes that enclose all primitives stored in any of the node's children. Axis-aligned bounding boxes (\acrshort{aabb}s) are the most popular volumes as they offer a good trade-off between intersection speed and tight fit around geometry. Bounding spheres offer the fastest intersection algorithm, but generally have larger volumes and oriented bounding boxes (\acrshort{obb}s) can provide the closest fit, but lack in intersection speed. When traversing such data structures, each child node needs to be checked as bounding volumes could overlap each other. A possible traversal algorithm is explained more detailed in section \ref{traversal}. In the rest of this section, basic \acrfull{bvh} construction algorithms are presented. 

A basic top-down algorithm starts at the root of the tree containing all scene primitives. The node is then split into two disjoint sets, which are set as children of the root. This process is repeated recursively for both children until some termination criteria is met and the given node is converted into a leaf. At each step, a bounding volume enclosing all primitives is assigned to the node. Given its exponential nature, splitting primitives into disjoint sets is a very complex problem. According to Popov et al.\cite{popov09harmful}, primitives can be split with a complexity of $O(n^6)$, which is not feasible in practice. As a result, primitives are generally split using axis-aligned planes. First, one splitting axis needs to be selected. Primitives are then ordered along that axis, most of the time according to their bounding volume centroid. Finding the split can then be done in three basic ways. A spatial median split cuts the bounding volume in half, object median uses a split with the same number of primitives in both halves and the most common approach splits primitives utilizing a cost model. 

The most common cost model is the \acrfull{sah}\cite{goldsmith_automatic_1987,macdonald_heuristics_1990}, which can be used to calculate the cost of a split as follows:
\[
    SAH(i)=S_L(i)p_L(i)+S_R(i)p_R(i)
\]
where $S_L(i)$ and $S_R(i)$ are the surface areas of the bounding boxes of the left and right subsets and $p_L(i)$ and $p_R(i)$ are the number of primitives in the subsets, respectively. Computing a full sweep SAH considering all primitives can be very expensive, thus a very popular method is binning SAH as proposed by Havran et al.\cite{havran06sahbin} and Wald et al.\cite{wald07fastConstruction}. In binning SAH primitives are projected into $b$ equally-spaced bins on which the SAH is then evaluated. 

Other cost functions include the occlusion heuristic\cite{vinkler12visibility} based on the assumed visibility of primitives and the ray distribution heuristic\cite{bittner09rdh}, which takes a sample the of the ray distribution into account. Both cost functions aim to improve the surface area heuristic, but might produce unstable results when used on their own, so it makes sense to combine those probabilities with the ones given by plain SAH.

Bottom-up construction\cite{Walter2008FastAC} starts by considering all primitives in clusters enclosed by bounding volumes. The two closest clusters are then merged to form a new node in the \acrshort{bvh} and the process is repeated until only one cluster is left, forming the trees root.
This approach is capable of producing trees with a better global cost compared to top-down approaches, which often only consider a local solution to cost functions. However, bottom-up construction in general is more expensive and the top levels might be poorly optimized due to the focus on lower levels.

Incremental construction\cite{goldsmith_automatic_1987} starts with an empty BVH and inserts primitives by traversing the tree and finding an appropriate leaf node. Once the number of primitives in a leaf gets too big, it is split into two new children. In general though, BVHs constructed this way are of lower quality making the approach less interesting. Nonetheless, it might still be useful if only parts of the input are available at the beginning, for example when streaming data.
\cleardoublepage
\section{Related Work}
\subsection{Denoising}
\label{denoising}
Efficient denoising techniques are essential to real-time path tracing, as only a limited number of samples-per-pixel is available for any given frame. While offline methods achieve the best quality, only interactive and real-time approaches are relevant in the context of this thesis. Yan et al.\cite{yan14denoising} proposed a sheared filtering approach that achieves interactive frame rates. Schied et al.\cite{schied_spatiotemporal_2017} proposed an approach that combines path tracing output and previous frame data with a noise free G-buffer generated using a rasterization pass to feed a wavelet filter. Mara et al.\cite{mara17towards} independently proposed a similar ray-tracing/rasterization hybrid method. They used a bilateral filter variant to achieve similar results. Chaitanya et al.\cite{chaitanya_interactive_2017} showed that neural networks can be used for denoising at interactive frame rates by using a convolutional neural network (CNN) to map noisy input images to noise-free output. In this approach, temporal noise was addressed by using recurrent connections in each layer of the CNN. Regression-based noise filtering produces higher quality output at the cost of more expensive computation. Koskela et al.\cite{koskela2019bmfr} were the first to implement a regression-based reconstruction pipeline that runs in real time. 

State-of-the-art denoising approaches are able to produce a denoised, temporally stable sequences of images using only one sample-per-pixel. However, denoising was not the focus of this work and will not be mentioned in the remainder of this thesis. Consequently, the implemented path tracer produces noisy one sample-per-pixel output leaving the choice of denoising technique open, even though applying any denoising technique would be an interesting topic for some future work.
\subsection{Space Subdivision}
Fuchs et al.\cite{fuchs1980bsp} proposed one of the first binary space partitioning trees, also referred to as kd-trees, which is built by recursively splitting the space along a given axis. This cut position is selected in a way that both sides contain a relatively equal number of objects. Glassner\cite{glassner_space_1984} described an approach for generating octrees that, for each recursive step, splits the given subspace at the spatial median along all three axis, resulting in eight new subregions. While trees with higher branching factors generally have a lower depth, binary trees allow for simpler traversal, as only a two-way decision is needed at each step. Kaplan\cite{kaplan_use_1985} expanded on Glassners idea by introducing a very similar implementation utilizing binary trees instead of octrees. Fujimoto et al.\cite{fujimoto_arts_1986}, while also using octrees, achieved a significant speed improvement by using incremental integer arithmetic to optimize the traversal algorithm. Havran and Bittner~\cite{Havran02onimproving} introduced additional traversal improvements utilizing a new termination criteria and a novel approach for clipping primitives. More modern kd-tree construction algorithms\cite{roccia2012kdtree,choi2010sahKdTree,wu2011sahKdTree} make use of the Surface Area Heuristic (SAH)\cite{goldsmith_automatic_1987,macdonald_heuristics_1990} further improving their performance. Li et al.\cite{li17parallelKD} proposed a construction algorithm based on Morton codes\cite{morton66curve} to enable a maximum level of parallelism. Hunt et al.\cite{hunt07lazybuild} proposed kd-tree construction from a given hierarchy. A similar approach for BVH construction is presented in section \ref{phr}
\subsection{Object Subdivision}
Bounding volume hierarchies were first mentioned by James Clark~\cite{clark1976bvh} and also referenced by Turner Whitted\cite{whitted_improved_1980}. Meister et al.\cite{meister21survey} published a report that reviews state-of-the-art BVH methods and discusses best practices. 

In the context of interactive and real-time rendering, construction speed is very crucial, especially when dealing with dynamic scenes. However, parallelizing the construction process is not straightforward. One parallel solution is a BVH based on Morton codes, which reduces the construction process to sorting primitives along the Morton curve\cite{morton66curve}. Sorting Morton codes with fixed length has a complexity of $O(n)$ and can be parallelized fairly efficiently. Such an approach was first proposed by Lauterbach et al.\cite{lauterbach09lbvh} as a top down GPU-based algorithm called \acrfull{lbvh}. A similar CPU based approach is elaborated further in section \ref{aux}. Pantaleoni and Luebke\cite{pantaleoni10hlbvh} proposed hierarchical LBVH, which combines LBVH with sweeping SAH in the upper levels of the tree and Garanzha et al.\cite{garanzha11hlbvh} applied binning SAH using Morton code prefixes as bin indices. Karras\cite{karras12lbvh} improved LBVH by using a special node layout and bottom-up reduction to construct the whole tree in parallel. Apetrei\cite{apetrei14lbvh} further improved the approach by constructing the tree and computing bounding boxes in one go, which was previously done in two seperate steps. Chitalu et al.\cite{chitalu20lbvh} combined LBVH with an ostensibly-implicit layout, which is the fastest construction algorithm to date\cite{meister21survey}.
Another improvement was presented by Vinkler et al.\cite{vinkler17morton} where Morton codes also encode the size of scene primitives. Hou et al.\cite{hou11bvh} proposed another GPU-base parallel algorithm for constructing kd-trees and BVHs by using partial breadth-first search and dumping results to CPU memory in between iterations to control GPU memory. 

While space subdivision approaches have previously been regarded as the best acceleration data structure~\cite{havrand2000comparison}, object subdivision has since caught up and overtaken~\cite{vinkler2015comparison}, making it the most popular approach for path tracing. Some of the advantages of bounding volume hierarchies include a predictable memory footprint, robust and efficient query and scalable construction. In addition, bounding volume hierarchies are very beneficial in dynamic scenes\cite{wald_ray_2007}, as they can be re-fit efficiently on scene changes. Because of these advantages, only object subdivision approaches will be considered in the following sections of this thesis.
\cleardoublepage
\section{Interactive Path Tracer}
This section provides a broad overview over the interactive path tracer written for this thesis ...

This section explains the BVH construction algorithm at the core of this thesis, first introduced by Hendrich et al.~\cite{hendrich_parallel_2017}, as well as the contributions made by this work. These contributions include applying the approach to an interactive path tracer and introducing two approaches for optimizing the hyperparameters utilized by the algorithm. 

\subsection{Project Environment}
The fundamental point of this thesis was to write and optimize a CPU path tracer. First and foremost, to see how well path tracing performs on CPUs compared to GPUs. In general, CPUs are designed to execute serial instructions on an intermediate amount of data very fast, while GPUs are optimized to process instructions in parallel using little memory but maximizing throughput. Consequently, CPUs consist of few powerful cores, use pipelining, branch prediction, out-of-order execution and utilize a bigger coherent cache while GPUs consist of many weaker cores and are optimized to run the graphics pipeline. This makes them perform very well with highly coherent work. Path tracing however, can often be very incoherent as rays might be scattered in all kinds of directions hitting very different primitives. In order to find ray intersections it is also necessary to have the whole acceleration structure in memory, which might be an issue with the limited GPU memory. Lastly, data transfer between CPU and GPU is costly and can be avoided by processing data directly on the CPU. 

Considering those aspects, it is an interesting topic implement and benchmark a CPU path tracer.

Another though was that, given sufficient performance, the CPU path tracer could be used in hybrid with GPUs to increase their performance. 

And finally, the goal was to keep all ideas and algorithms as general as possible and not bound to a specific implementation.

The path tracer is written in the programming language Go, which has a few advantages and disadvantages. One of the main reasons for using Go was the built in thread management in the form of go routines. % TODO: Explain go routines
In addition to keeping concurrency efficient, this keeps the source code very readable tieing in very well with the overall readability of go. Another bonus the language offeres is great benchmarking support as part of the language.

The above listed advantages were enough to choose the language, however, there are still a few shortcomings. The main issue is the performance of the language. Even though Go allows to write code somewhat close to the system, it still is a rather high level language. The built in garbage collector might be very well optimized, but having a garbage collector at all is a substential performance sink and the Go compiler favors compile over execution speed. Some of those problems can be optimized, for example the whole path tracer was written with a zero alloc approach, reusing structs and using C style pointer parameter returns, but other languages like C or Rust could have performed favorably.

\subsection{Path Tracer}
The renderer itself supports two types of primitives. Spheres, as they have the simplest intersection function and triangles, 
% TODO: Mention intersection algorithms
as most geometry can be represented or at least approximated using triangles. Each primitive can have an associated material. Those are currently limited to three different ones, which all have an albedo color value and differ in the way they scatter incoming rays. Diffuse materials scatter rays in a random direction within a unit sphere. Reflective materials reflect the ray in the reflection direction and a certain random deviation is added depending on the diffussion coefficient. Refractive materials utilize snell's law to represent glass objects and other dielectrics. In addition to those materials, there are also light sources, which instead of scattering rays add color to them. 

% TODO: Check if that's actually the way in the final project
Rendering a frame is done line by line utilizing the worker pattern. $k$ threads render a single line and then wait for a new 
% TODO: Mark code in text?
line using Go channels. A pixel is colored by casting a new ray with a slightly offset origin within said pixel and a direction directed towards the according pixel in the image plane. The closest intersection is determined by traversing the 
% TODO: Add see section
scene's BVH (See section ...) returning a hit record containing relevant information like intersection point, normal and material at said point. If no intersection is found, the pixel is colored in the color specified in the miss shader, otherwise the closest hit shader will be called with the intersection information. The closest hit shader casts a new ray from the intersection point depending on how the material at given point scatters and calls itself recursively if a new intersection is found and the maximum depth has not been reached. At each step, emitted light will be added to the pixel and multiplied by the materials albedo. Each thread reuses a single ray and hit structure to avoid allocating additional memory.
\subsection{Bounding Volume Hierarchy}
\subsubsection{Construction}
Bounding volume hierarchies are constructed using the algorithm called PHR as proposed by Hendrich et al.\cite{hendrich_parallel_2017}. As previously established, applying full sweep SAH to a whole scene is magnitudes too expensive. PHR tackles this problem by first constructing an auxiliary BVH, which then serves as a hierarchy to find much smaller sets of nodes. Those smaller sets can then be split failry inexpensive by applying full sweep SAH. The two resulting cuts are then refined, meaning that some of the nodes within those cuts are replaced by their children to keep cuts at a desired size.



\subsubsection{Traversal}
\section{Results}
\label{results}
Both the BVH builder and path tracer were implemented in Go and only utilize the CPU. Code is optimized moderately without exploiting any SIMD instructions. A series of tests was conducted to compare the build times, ray tracing performance and resulting time per frame rendered between different hyperparameter configurations. LBVH was used as reference, PHR-Fast and PHR-HQ used the parameters proposed by Hendrich et al.\cite{hendrich_parallel_2017}, namely $\alpha=0.5, \delta=6$ and $\alpha=0.55, \delta=9$, respectively. PHR-Grid uses parameters based on the proposed grid search approach over the search space $\alpha\in\{0.4,0.45,0.5.0.55\}, \delta\in\{6,7,8,9\}$ and PHR-BO uses parameters resulting from a Bayesian optimization over the equivalent interval $\alpha\in[0.4,0.55], \delta\in[6,9]$. The Bayesian optimization itself is executed utilizing the bo framework\cite{ou19bo} and is based on a Gaussian process and expected improvement as exploration strategy. 

To make results more reliable, all numbers were averaged over ten executions using the bench framework\cite{ou20bench}. Furthermore, the CPU, an AMD Ryzen 2600 eight core processor with 3.4 GHz, was locked to 90 percent capacity to prevent irregularities due to overheating or other high performance fluctuations. Note that the deviation percentages are left out for clarity in the tables presented in this thesis, but the full results are available in the attached files. 

Render times are also averaged over three representative views for each scene. To keep times in an interactive window, only the relatively small resolutions 256x256 and 512x512 were tested with one sample per pixel. 
\subsection{Multi-Bounding Volume Hierarchies}
\label{multi_bvh}
As mentioned in section \ref{phr_algorithm}, the PHR algorithm allows the construction of bounding volume hierarchies with higher branching factors, which is especially useful for SIMD path tracers. Even though the evaluated path tracer does not utilize any SIMD instructions, I compared the build and trace performance of different multi-BVHs. As expected, the performance difference between branching factors was insignificant in most cases. 4-ary BVHs had slightly faster trace times, while 16-ary BVH construction was slightly slower. The following tests were all performed on 2-ary BVHs. 
\subsection{Frame Performance}
The main part of the experiment was about comparing the resulting frame times of all configurations. Table \ref{tab:frametime} shows build time, SAH cost, average render time over the compared view points and the resulting frame times. Note that the PHR build times do not include construction of the auxiliary bounding volume hierarchy, as those would be reused over several frames. 

First of all, the numbers clearly show the impact different PHR parameters can have on the build and trace time of the algorithm. PHR-Fast was indeed fairly fast, but the achieved trace speed is even below the baseline, LBVH, in some cases. PHR-HQ had the fastest trace speed in most cases, but the high build duration often leads to higher frame times. A noticeable difference can be seen between the different scene types. PHR performed comparably worse in single object scenes like Bunny, Dragon and Happy Buddha, often not exceeding the render performance of LBVH by much. Sibenik's and Sponza's render times on the other hand, were improved more significantly by applying PHR. Finally, PHR-Grid and PHR-BO were able to improve frame times significantly in a number of cases. However, Bayesian optimization delivered rather inconsistent results and no approach was able to find the optimal parameters in every case. This is probably a result of overfitting the evaluation function and their parameters to certain scenes. Nonetheless, both PHR-Grid and PHR-BO are able to improve frame times compared to PHR-Fast and PHR-HQ. Figure \ref{fig:difference} shows the average relative performance compared to LBVH in percent. While PHR-Fast only improves frame times by 33\% on average, both optimization approaches surpass an average improvement of 50\%.
\begin{figure} [H]
    \centering
    \includegraphics[width=300pt]{images/performance_difference.pdf}
    \caption{Relative difference to LBVH in percent.}
    \label{fig:difference}
\end{figure}
\subsection{Optimization Performance}
Figure \ref{fig:optimization} shows optimization times divided through average frame time, or in other words, how many frames could potentially be rendered instead of executing the optimization. Considering that the average performance increase of both methods amounted around 50\%, i.e. can potentially half the rendering time, optimizations start to become viable at half a frames duration and below. So even though the search space used in grid search is relatively low, the achieved times never reached what would be feasible in interactive applications. Bayesian optimization performed considerably better, but only reached competitive times in a few cases. Note that reaching this time does not equal a performance increase but just a hypothetical chance that frame times are increased. 

This shows that the presented optimization approaches still lack in performance and are not yet viable. PHR needs to be executed to evaluate the cost function, which is especially costly for parameters that result in high build times. This could be improved by limiting the search spacer further or making it dynamic and related to the scene's complexity. An interrupt after a maximum execution time might also be a solution. Bayesian optimization uses a costly run with maxed out parameters to determine the maximum build cost. This could be solved by reusing max values from previous optimizations. This topic is discussed further in section \ref{discussion_execute}.
\begin{figure}[H]
    \centering
    \includegraphics[width=300pt]{images/frame_per_optimization.pdf}
    \caption{Number of potential frames during optimization.}
    \label{fig:optimization}
\end{figure}
\clearpage
\begin{table}
\caption{Performance comparison of a representative selection of tested configuration at 256x256 resolution.}
\label{tab:frametime}
\centering
\begin{tabular}{ | c | m{3.5em} | m{3.5em} | m{3.5em} | m{3.5em} | m{3.5em} |  m{3.5em}|}
\hline
& build time (ms) & render time (ms) & frame time ms & build time (ms) & render time (ms) & frame time ms\\
\hline
& \multicolumn{1}{|m{4.5em}}{\includegraphics[width=60pt]{images/bunny.png}} &     \multicolumn{2}{m{4em}|}{Bunny \#triangles 144k} 
& \multicolumn{1}{|m{4.5em}}{\includegraphics[width=60pt]{images/sponza.png}} & \multicolumn{2}{m{4em}|}{Sponza \#triangles 66k}\\
\hline
LBVH & 152.0 & 21.9 & 173.9            & 70.4 & 227.0 & 297.5 \\
PHR-Fast & 70.5 & 20.1 & 90.6          & 27.6 & 184 & 211.6 \\
PHR-HQ & 273.0 & 20.7 & 293.7          &  94.7 & 182 & 276.7\\
\hline
PHR-Grid & 14.4 & 20.8 & 35.2        &  17.2 & 207 & 224.2\\
PHR-BO & 14.2 & 20.7 & 34.97         &  21.7 & 200 & 221.7\\
\hline
& \multicolumn{1}{|m{4.5em}}{\includegraphics[width=60pt]{images/dragon.png}} &     \multicolumn{2}{m{4em}|}{Dragon \#triangles 817k}

& \multicolumn{1}{|m{4.5em}}{\includegraphics[width=60pt]{images/sibenik.png}} & \multicolumn{2}{m{4em}|}{Sibenik \#triangles 75k}\\

\hline
LBVH & 900.0 & 28.1 & 928.2              &  72.1 & 226.1 & 298.1\\
PHR-Fast & 150.0 & 32.5 & 182.5            &  32.5 & 205 & 237.5\\
PHR-HQ & 1090.0 & 29.0 & 1119.0              &  98.4 & 187  &285.4\\
\hline
PHR-Grid & 31.7 & 63.5 & 95.2            & 19.1 & 236 & 255.1  \\
PHR-BO & 30.8 & 66.1 & 96.9              & 28.1 & 220 & 248.1 \\
\hline
& \multicolumn{1}{|m{4.5em}}{\includegraphics[width=60pt]{images/buddha.png}} & \multicolumn{2}{m{4em}|}{Happy Buddha \#triangles 1087k}

& \multicolumn{1}{|m{4.5em}}{\includegraphics[width=60pt]{images/san_miguel.png}} &     \multicolumn{2}{m{4.5em}|}{San Miguel \#triangles 5617k}\\


\hline
LBVH & 1080 & 25 & 1105                    & 6010.0 & 816.0 & 6826.6 \\                   
PHR-Fast & 190 & 27.6 & 217.6                   & 432.0 & 7830.0 & 8262.0 \\
PHR-HQ & 1250 & 23.7 & 1273.7                     & 2600.0 & 812.3 & 3412.3 \\
\hline
PHR-Grid & 40.5 & 57.5 &98                   & 1300.0 & 1856.6 & 3156.6  \\
PHR-BO & 42.1 &58 &100.1                     & 815.0 & 3203.3 & 4018.3 \\
\hline
\end{tabular}
\end{table}
\cleardoublepage
\section{Discussion}

% TODO: Results slightly different, than in original paper... reason: no triangle quartets used in my implementation
\section{Conclusion}

\subsection{Further Work}
Path tracing is a very extensive topic and especially considering the task of writing a complete path tracer, there is a lot of work left open. Considering the path tracer, a missing but essential aspect is denoising. As discussed earlier, recently a lot of progress has been achieved in that field of research and integrating some of that into this project would be an interesting addition. Importance sampling is another technique that was not considered in this work, but might improve it a decent amount. 
Staying with the main focus of this thesis, replacing LBVH with another fast BVH builder could bring interesting results. The full sweep SAH at the core of PHR is still relatively expensive and might be replacable by the more efficient binning SAH for better performance, or extended by using other cost functions like ray distribution heuristic or occlusion heuristic. Finally, optimization of PHRs hyperparameters is not ideal yet and might be improved by either using other optimization procedures, or improving the evaluation function.
\makeglossaries
\newglossaryentry{point}
{
    name=Point
    description={A point $P$ in three-dimensional space, consisting of three components x, y and z. In this thesis denoted using capital letters}
}
\newglossaryentry{vector}
{
    name=Vector
    description={A vector $v$ in three-dimensional space having a direction and a magnitude. In this thesis denoted using lower case letters}
}
\newglossaryentry{ray}
{
    name=Ray
    description={A ray is a semi-infinite line, usually in three-dimensional space. In this work the common representation using an origin $O$ and a direction $d$ is used. 
    \[P(t)=O+td\]
    Where the $O$ is some point within the scene space and the $d$ is some vector along which the ray travels in a straight line. Points along this line can be described using the distance $t$, with $P(0)=O$ and $P(1)=O+d$.}
}
\newglossaryentry{ray casting}
{
    name=Ray Casting
    description={}
}
\newglossaryentry{tree}
{
    name=Tree
    description={}
}
\newacronym{aabb}{AABB}{Axis-Aligned Bounding Box}
\newacronym{obb}{OBB}{Oriented Bounding Box}
\newacronym{bvh}{BVH}{Bounding Volume Hierarchy}
\newacronym{phr}{PHR}{Progressive Hierarchical Refinement}
\newglossaryentry{aabb}
{
    name=Axis-aligned Bounding Box
    description={}
}
\newglossaryentry{obb}
{
    name=Oriented Bounding Box
    description={}
}
\newglossaryentry{bounding sphere}
{
    name=Bounding Sphere
    description={}
}

\section{Terminology}
This section provides information and notation about terms used throughout this thesis.
\printglossary[type=\acronymtype]
\printglossary

\input{literatures/bib.tex}
\end{document}