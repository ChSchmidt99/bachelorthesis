\section{Conclusion and Future Work}
\label{conclusion}
Given the extensive nature of the topic, a lot of work is left open. Regarding the path tracer, incorporating any sort of denoising is a necessary step to achieving usable results. The performance is not quite at a sufficient point, so finding additional optimizations is another open topic. As mentioned before, Go is not an ideal language for such a performance critical task, so translating the project into a better fit language would be interesting as well. Importance sampling is another technique that was not considered in this work, but might improve the results by a decent amount. 

The implementation progressive hierarchical refinement could be improved further by optimizing construction of auxiliary BVHs, as this part still lacks in performance. The full sweep SAH evaluation at the core of the algorithm is relatively expensive, so replacing it by binning SAH might be advantageous for interactive and real-time applications. Evaluating the influence of other cost functions like the RDH to both the split function and the parameter cost evaluation might lead to interesting results. SIMD instructions were not used in any part of this project, so applying those where appropriate could improve the path tracer significantly.
\\\\
Path tracing is an essential part in computer graphics and recent advancements in real-time path tracing pushed its popularity even further. This thesis presented an overview over the basics of path tracing and how an interactive path tracer might be implemented based on an example written for CPUs in the programming language Go. The evaluation of this path tracer showed that interactive frame rates can be achieved for reasonably small scenes and resolutions, but competitive usage requires additional work. 

Furthermore, the state-of-the-art algorithm called progressive hierarchical refinement was presented and extended by a novel optimization technique for the used hyperparameters. This method was tested extensively and its potential performance increase was validated.
\cleardoublepage