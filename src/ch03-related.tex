\section{Related Work}
\subsection{Denoising}
\label{denoising}
Efficient denoising techniques are essential to real-time path tracing, as only a limited number of samples-per-pixel is available for any given frame. While offline methods achieve the best quality, only interactive and real-time approaches are relevant in the context of this thesis. Yan et al.\cite{yan14denoising} proposed a sheared filtering approach that achieves interactive frame rates. Schied et al.\cite{schied_spatiotemporal_2017} proposed an approach that combines path tracing output and previous frame data with a noise free G-buffer generated using a rasterization pass to feed a wavelet filter. Mara et al.\cite{mara17towards} independently proposed a similar ray-tracing/rasterization hybrid method. They used a bilateral filter variant to achieve similar results. Chaitanya et al.\cite{chaitanya_interactive_2017} showed that neural networks can be used for denoising at interactive frame rates by using a convolutional neural network (CNN) to map noisy input images to noise-free output. In this approach, temporal noise was addressed by using recurrent connections in each layer of the CNN. Regression-based noise filtering produces higher quality output at the cost of more expensive computation. Koskela et al.\cite{koskela2019bmfr} were the first to implement a regression-based reconstruction pipeline that runs in real time. 

State-of-the-art denoising approaches are able to produce a denoised, temporally stable sequences of images using only one sample-per-pixel. However, denoising was not the focus of this work and will not be mentioned in the remainder of this thesis. Consequently, the implemented path tracer produces noisy one sample-per-pixel output leaving the choice of denoising technique open, even though applying any denoising technique would be an interesting topic for some future work.
\subsection{Space Subdivision}
Fuchs et al.\cite{fuchs1980bsp} proposed one of the first binary space partitioning trees, also referred to as kd-trees, which is built by recursively splitting the space along a given axis. This cut position is selected in a way that both sides contain a relatively equal number of objects. Glassner\cite{glassner_space_1984} described an approach for generating octrees that, for each recursive step, splits the given subspace at the spatial median along all three axis, resulting in eight new subregions. While trees with higher branching factors generally have a lower depth, binary trees allow for simpler traversal, as only a two-way decision is needed at each step. Kaplan\cite{kaplan_use_1985} expanded on Glassners idea by introducing a very similar implementation utilizing binary trees instead of octrees. Fujimoto et al.\cite{fujimoto_arts_1986}, while also using octrees, achieved a significant speed improvement by using incremental integer arithmetic to optimize the traversal algorithm. Havran and Bittner~\cite{Havran02onimproving} introduced additional traversal improvements utilizing a new termination criteria and a novel approach for clipping primitives. More modern kd-tree construction algorithms\cite{roccia2012kdtree,choi2010sahKdTree,wu2011sahKdTree} make use of the Surface Area Heuristic (SAH)\cite{goldsmith_automatic_1987,macdonald_heuristics_1990} further improving their performance. Li et al.\cite{li17parallelKD} proposed a construction algorithm based on Morton codes\cite{morton66curve} to enable a maximum level of parallelism. Hunt et al.\cite{hunt07lazybuild} proposed kd-tree construction from a given hierarchy. A similar approach for BVH construction is presented in section \ref{phr}
\subsection{Object Subdivision}
Bounding volume hierarchies were first mentioned by James Clark~\cite{clark1976bvh} and also referenced by Turner Whitted\cite{whitted_improved_1980}. Meister et al.\cite{meister21survey} published a report that reviews state-of-the-art BVH methods and discusses best practices. 

In the context of interactive and real-time rendering, construction speed is very crucial, especially when dealing with dynamic scenes. However, parallelizing the construction process is not straightforward. One parallel solution is a BVH based on Morton codes, which reduces the construction process to sorting primitives along the Morton curve\cite{morton66curve}. Sorting Morton codes with fixed length has a complexity of $O(n)$ and can be parallelized fairly efficiently. Such an approach was first proposed by Lauterbach et al.\cite{lauterbach09lbvh} as a top down GPU-based algorithm called \acrfull{lbvh}. A similar CPU based approach is elaborated further in section \ref{aux}. Pantaleoni and Luebke\cite{pantaleoni10hlbvh} proposed hierarchical LBVH, which combines LBVH with sweeping SAH in the upper levels of the tree and Garanzha et al.\cite{garanzha11hlbvh} applied binning SAH using Morton code prefixes as bin indices. Karras\cite{karras12lbvh} improved LBVH by using a special node layout and bottom-up reduction to construct the whole tree in parallel. Apetrei\cite{apetrei14lbvh} further improved the approach by constructing the tree and computing bounding boxes in one go, which was previously done in two seperate steps. Chitalu et al.\cite{chitalu20lbvh} combined LBVH with an ostensibly-implicit layout, which is the fastest construction algorithm to date\cite{meister21survey}.
Another improvement was presented by Vinkler et al.\cite{vinkler17morton} where Morton codes also encode the size of scene primitives. Hou et al.\cite{hou11bvh} proposed another GPU-base parallel algorithm for constructing kd-trees and BVHs by using partial breadth-first search and dumping results to CPU memory in between iterations to control GPU memory. 

While space subdivision approaches have previously been regarded as the best acceleration data structure~\cite{havrand2000comparison}, object subdivision has since caught up and overtaken~\cite{vinkler2015comparison}, making it the most popular approach for path tracing. Some of the advantages of bounding volume hierarchies include a predictable memory footprint, robust and efficient query and scalable construction. In addition, bounding volume hierarchies are very beneficial in dynamic scenes\cite{wald_ray_2007}, as they can be re-fit efficiently on scene changes. Because of these advantages, only object subdivision approaches will be considered in the following sections of this thesis.
\cleardoublepage